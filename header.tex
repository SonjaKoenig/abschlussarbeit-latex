% Fonts {{{
\usepackage{helvet}             % Helvetica (eig. Nimbus Sans) Klon
\usepackage[sc]{mathpazo}       % Serif Font für Texte
\usepackage{microtype}          % Besseres Kerning, weniger Trennungen
% }}}

% Packages {{{
\usepackage{graphicx}
\usepackage{amsmath, amsthm, amssymb}
\usepackage{listings}
\usepackage{multicol}
\usepackage{booktabs}
\usepackage{url}
\usepackage{ccicons}

\usepackage{natbib}
\setcitestyle{square,numbers,comma,sort}
% }}}

% Bildunterschriften {{{
\usepackage{caption}   % Sans-Serif Font für Bildunterschriften
\captionsetup{labelfont={sf,bf},font=sf}
\usepackage[sf,bf,SF]{subfigure}
% }}}

% Bilder {{{
\usepackage{tikz}
% }}}

% Satzspiegel {{{
\renewcommand{\baselinestretch}{1.10}
\setlength{\parindent}{0pt}
\setlength{\parskip}{\baselineskip}
\widowpenalty10000
\clubpenalty10000
% }}}

% Überschriftstiefe im TOC
\setcounter{tocdepth}{2}

% Literaturverzeichnis
\bibliographystyle{dinat}

% Kopf- und Fußzeilen {{{
\usepackage{scrpage2}
\pagestyle{scrheadings}
\clearscrheadfoot
\setkomafont{pagehead}{\sffamily}
\automark[section]{chapter}
\ohead{\headmark}
\ofoot{\pagemark}
\setheadsepline{.4pt}
% }}}

% Titelseite {{{
\newlength{\titlewidth}
\newlength{\titleoverlap}

\setlength{\titleoverlap}{.5in}

\setlength{\titlewidth}{2\titleoverlap}
\addtolength{\titlewidth}{\textwidth}

\setkomafont{title}{\sffamily\huge}

\lowertitleback{
\begin{tabbing}
    \hspace*{12em} \= \kill

    Fassung vom: \> \today\\
    \\
    \parbox{10em}{%
    \ccbyncnd
    }
    \> \parbox[t]{32em}{\raggedright%
    Diese Arbeit ist lizensiert unter der Creative Commons
    Namensnennung-Keine kommerzielle Nutzung-Keine Bearbeitung 3.0 Deutschland 
    Lizenz.\\[.5em]
    Nähere Informationen finden Sie unter:
    \url{http://creativecommons.org/licenses/by-nc-nd/3.0/de/}\\
    oder senden Sie einen Brief an: Creative Commons, 171 Second Street, Suite 300, San
    Francisco, California, 94105, USA.
    }\\
\end{tabbing}
}

% PDF Eigenschaften {{{
\makeatletter
\usepackage[pdftex,
	unicode=true,
	colorlinks=true,
	linkcolor=black,
	citecolor=black,
	urlcolor=black,
	pdfauthor={\@author}
]{hyperref}
\makeatother
% }}}

% Nützliche Befehle {{{
\newcommand{\etal}{\emph{et\,al.}}
\newtheorem{defn}{Definition}[chapter]
\newtheorem{thm}{Satz}[section]
% }}}

\hyphenation{%
Fahr-zeug-zu-In-fra-struk-tur-Kom-mu-ni-ka-ti-on
}
